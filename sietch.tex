% Dedicated to Craig Leto
\documentclass{article}
\RequirePackage{amsmath}
\RequirePackage{bytefield}
\RequirePackage{graphicx}
\RequirePackage{newtxmath}
\RequirePackage{mathtools}
\RequirePackage{xspace}
\RequirePackage{url}
\RequirePackage{changepage}
\RequirePackage{enumitem}
\RequirePackage{tabularx}
\RequirePackage{hhline}
\RequirePackage[usestackEOL]{stackengine}
\RequirePackage{comment}
\RequirePackage{needspace}
\RequirePackage[nobottomtitles]{titlesec}
\RequirePackage[hang]{footmisc}
\RequirePackage{xstring}
\RequirePackage[unicode,bookmarksnumbered,bookmarksopen,pdfview=Fit]{hyperref}
\RequirePackage{cleveref}
\RequirePackage{nameref}

\RequirePackage[style=alphabetic,maxbibnames=99,dateabbrev=false,urldate=iso8601,backref=true,backrefstyle=none,backend=biber]{biblatex}
\addbibresource{sietch.bib}

% Fonts
\RequirePackage{lmodern}
\RequirePackage{quattrocento}
\RequirePackage[bb=ams]{mathalfa}

% Quattrocento is beautiful but doesn't have an italic face. So we scale
% New Century Schoolbook italic to fit in with slanted Quattrocento and
% match its x height.
\renewcommand{\emph}[1]{\hspace{0.15em}{\fontfamily{pnc}\selectfont\scalebox{1.02}[0.999]{\textit{#1}}}\hspace{0.02em}}

% While we're at it, let's match the tt x height to Quattrocento as well.
\let\oldtexttt\texttt
\let\oldmathtt\mathtt
\renewcommand{\texttt}[1]{\scalebox{1.02}[1.07]{\oldtexttt{#1}}}
\renewcommand{\mathtt}[1]{\scalebox{1.02}[1.07]{$\oldmathtt{#1}$}}

\newcommand{\zsendmany}{\textbf{z\_sendmany}}

% bold but not extended
\newcommand{\textbnx}[1]{{\fontseries{b}\selectfont #1}}


\crefformat{footnote}{#2\footnotemark[#1]#3}

\DeclareLabelalphaTemplate{
  \labelelement{\field{citekey}}
}

\DefineBibliographyStrings{english}{
  page  = {page},
  pages = {pages},
  backrefpage = {\mbox{$\uparrow$ p\!}},
  backrefpages = {\mbox{$\uparrow$ p\!}}
}

\setlength{\oddsidemargin}{-0.25in}
\setlength{\textwidth}{7in}
\setlength{\topmargin}{-0.75in}
\setlength{\textheight}{9.2in}
\setlength{\parindent}{0ex}
\renewcommand{\arraystretch}{1.4}
\overfullrule=2cm

\setlength{\footnotemargin}{0.6em}
\setlength{\footnotesep}{2ex}
\addtolength{\skip\footins}{3ex}

\renewcommand{\bottomtitlespace}{8ex}

% Use rubber lengths between paragraphs to improve default pagination.
% https://tex.stackexchange.com/questions/17178/vertical-spacing-pagination-and-ideal-results
\setlength{\parskip}{1.5ex plus 1pt minus 1pt}

\setlist[enumerate]{before=\vspace{-1ex}}
\setlist[itemize]{itemsep=0.5ex,topsep=0.2ex,before=\vspace{-1ex},after=\vspace{1.5ex}}

\newlist{formulae}{itemize}{3}
\setlist[formulae]{itemsep=0.2ex,topsep=0ex,leftmargin=1.5em,label=,after=\vspace{1.5ex}}

\newcommand{\docversion}{Whitepaper Version 0.1}
\newcommand{\termbf}[1]{\textbf{#1}\xspace}
\newcommand{\Hushlist}{\termbf{HushList}}
\newcommand{\HushList}{\termbf{HushList}}
\newcommand{\Hushlists}{\termbf{HushLists}}
\newcommand{\HushLists}{\termbf{HushLists}}

\newcommand{\doctitle}{Attacking Zcash Protocol For Fun And Profit}
\newcommand{\leadauthor}{Duke Leto + The Hush Developers}

\newcommand{\keywords}{anonymity, zcash protocol, cryptographic protocols, zk-SNARKs, metadata leakage, de-anonymization,\
electronic commerce and payment, financial privacy, zero knowledge mathematics,\
linkability, transaction graphs, shielded transactions, blockchain analysis }

\hypersetup{
  pdfborderstyle={/S/U/W 0.7},
  pdfinfo={
    Title={\doctitle, \docversion},
    Author={\leadauthor},
    Keywords={\keywords}
  }
}

\makeatletter
\renewcommand*{\@fnsymbol}[1]{\ensuremath{\ifcase#1\or \dagger\or \ddagger\or
    \mathsection\or \mathparagraph\else\@ctrerr\fi}}
\makeatother

\renewcommand{\sectionautorefname}{\S\!}
\renewcommand{\subsectionautorefname}{\S\!}
\renewcommand{\subsubsectionautorefname}{\S\!}
\renewcommand{\subparagraphautorefname}{\S\!}
\newcommand{\crossref}[1]{\autoref{#1}\, \emph{`\nameref*{#1}\kern -0.05em'} on p.\,\pageref*{#1}}

\newcommand{\nstrut}[1]{\texorpdfstring{#1\rule[-.2\baselineskip]{0pt}{\baselineskip}}{#1}}
\newcommand{\nsection}[1]{\section{\nstrut{#1}}}
\newcommand{\nsubsection}[1]{\subsection{\nstrut{#1}}}
\newcommand{\nsubsubsection}[1]{\subsubsection{\nstrut{#1}}}

\newcommand{\introlist}{\needspace{15ex}}
\newcommand{\introsection}{\needspace{30ex}}

\mathchardef\mhyphen="2D

% http://tex.stackexchange.com/a/309445/78411
\DeclareFontFamily{U}{FdSymbolA}{}
\DeclareFontShape{U}{FdSymbolA}{m}{n}{
    <-> s*[.4] FdSymbolA-Regular
}{}
\DeclareSymbolFont{fdsymbol}{U}{FdSymbolA}{m}{n}
\DeclareMathSymbol{\smallcirc}{\mathord}{fdsymbol}{"60}

\makeatletter
\newcommand{\hollowcolon}{\mathpalette\hollow@colon\relax}
\newcommand{\hollow@colon}[2]{
  \mspace{0.7mu}
  \vbox{\hbox{$\m@th#1\smallcirc$}\nointerlineskip\kern.45ex \hbox{$\m@th#1\smallcirc$}\kern-.06ex}
  \mspace{1mu}
}
\makeatother
\newcommand{\typecolon}{\;\hollowcolon\;}

% We just want one ampersand symbol from boisik.
\DeclareSymbolFont{bskadd}{U}{bskma}{m}{n}
\DeclareFontFamily{U}{bskma}{\skewchar\font130 }
\DeclareFontShape{U}{bskma}{m}{n}{<->bskma10}{}
\DeclareMathSymbol{\binampersand}{\mathbin}{bskadd}{"EE}

\newcommand{\hairspace}{~\!}
\newcommand{\hparen}{\hphantom{(}}

\newcommand{\hfrac}[2]{\scalebox{0.8}{$\genfrac{}{}{0.5pt}{0}{#1}{#2}$}}


\RequirePackage[usenames,dvipsnames]{xcolor}
% https://en.wikibooks.org/wiki/LaTeX/Colors#The_68_standard_colors_known_to_dvips
\newcommand{\todo}[1]{{\color{Sepia}\sf{TODO: #1}}}

\newcommand{\changedcolor}{magenta}
\newcommand{\setchanged}{\color{\changedcolor}}
\newcommand{\changed}[1]{\texorpdfstring{{\setchanged{#1}}}{#1}}

% terminology

\newcommand{\term}[1]{\textsl{#1}\kern 0.05em\xspace}
\newcommand{\titleterm}[1]{#1}
\newcommand{\quotedterm}[1]{``~\!\!\term{#1}''}
\newcommand{\conformance}[1]{\textbnx{#1}\xspace}

\newcommand{\Zcash}{\termbf{Zcash}}
\newcommand{\Hush}{\termbf{Hush}}
\newcommand{\Zerocash}{\termbf{Zerocash}}
\newcommand{\Bitcoin}{\termbf{Bitcoin}}
\newcommand{\CryptoNote}{\termbf{CryptoNote}}
\newcommand{\ZEC}{\termbf{ZEC}}
\newcommand{\ZEN}{\termbf{ZEN}}
\newcommand{\ZCL}{\termbf{ZCL}}
\newcommand{\KMD}{\termbf{KMD}}
\newcommand{\BTCH}{\termbf{BTCH}}
\newcommand{\BTCP}{\termbf{BTCP}}
\newcommand{\ZAU}{\termbf{ZAU}}
\newcommand{\VOT}{\termbf{VOT}}
\newcommand{\BTCZ}{\termbf{BTCZ}}
\newcommand{\LTZ}{\termbf{LTZ}}
\newcommand{\HUSH}{\termbf{HUSH}}
\newcommand{\zatoshi}{\term{zatoshi}}
\newcommand{\puposhi}{\term{puposhi}}
\newcommand{\zcashd}{\textsf{zcashd}\,}
\newcommand{\hushd}{\textsf{hushd}\,}

\newcommand{\MUST}{\conformance{MUST}}
\newcommand{\MUSTNOT}{\conformance{MUST NOT}}
\newcommand{\SHOULD}{\conformance{SHOULD}}
\newcommand{\SHOULDNOT}{\conformance{SHOULD NOT}}
\newcommand{\ALLCAPS}{\conformance{ALL CAPS}}

\newcommand{\note}{\term{note}}
\newcommand{\notes}{\term{notes}}
\newcommand{\Note}{\titleterm{Note}}
\newcommand{\Notes}{\titleterm{Notes}}
\newcommand{\dummy}{\term{dummy}}
\newcommand{\dummyNotes}{\term{dummy notes}}
\newcommand{\DummyNotes}{\titleterm{Dummy Notes}}
\newcommand{\commitmentScheme}{\term{commitment scheme}}
\newcommand{\commitmentTrapdoor}{\term{commitment trapdoor}}
\newcommand{\commitmentTrapdoors}{\term{commitment trapdoors}}
\newcommand{\trapdoor}{\term{trapdoor}}
\newcommand{\noteCommitment}{\term{note commitment}}
\newcommand{\noteCommitments}{\term{note commitments}}
\newcommand{\NoteCommitment}{\titleterm{Note Commitment}}
\newcommand{\NoteCommitments}{\titleterm{Note Commitments}}
\newcommand{\noteCommitmentTree}{\term{note commitment tree}}
\newcommand{\NoteCommitmentTree}{\titleterm{Note Commitment Tree}}
\newcommand{\noteTraceabilitySet}{\term{note traceability set}}
\newcommand{\noteTraceabilitySets}{\term{note traceability sets}}
\newcommand{\joinSplitDescription}{\term{JoinSplit description}}
\newcommand{\joinSplitDescriptions}{\term{JoinSplit descriptions}}
\newcommand{\JoinSplitDescriptions}{\titleterm{JoinSplit Descriptions}}
\newcommand{\sequenceOfJoinSplitDescriptions}{\changed{sequence of} \joinSplitDescription\changed{\term{s}}\xspace}
\newcommand{\joinSplitTransfer}{\term{JoinSplit transfer}}
\newcommand{\joinSplitTransfers}{\term{JoinSplit transfers}}
\newcommand{\JoinSplitTransfer}{\titleterm{JoinSplit Transfer}}
\newcommand{\JoinSplitTransfers}{\titleterm{JoinSplit Transfers}}
\newcommand{\joinSplitSignature}{\term{JoinSplit signature}}
\newcommand{\joinSplitSignatures}{\term{JoinSplit signatures}}
\newcommand{\joinSplitSigningKey}{\term{JoinSplit signing key}}
\newcommand{\joinSplitVerifyingKey}{\term{JoinSplit verifying key}}
\newcommand{\joinSplitStatement}{\term{JoinSplit statement}}
\newcommand{\joinSplitStatements}{\term{JoinSplit statements}}
\newcommand{\JoinSplitStatement}{\titleterm{JoinSplit Statement}}
\newcommand{\joinSplitProof}{\term{JoinSplit proof}}
\newcommand{\statement}{\term{statement}}
\newcommand{\zeroKnowledgeProof}{\term{zero-knowledge proof}}
\newcommand{\ZeroKnowledgeProofs}{\titleterm{Zero-Knowledge Proofs}}
\newcommand{\provingSystem}{\term{proving system}}
\newcommand{\zeroKnowledgeProvingSystem}{\term{zero-knowledge proving system}}
\newcommand{\ZeroKnowledgeProvingSystem}{\titleterm{Zero-Knowledge Proving System}}
\newcommand{\ppzkSNARK}{\term{preprocessing zk-SNARK}}
\newcommand{\provingKey}{\term{proving key}}
\newcommand{\zkProvingKeys}{\term{zero-knowledge proving keys}}
\newcommand{\verifyingKey}{\term{verifying key}}
\newcommand{\zkVerifyingKeys}{\term{zero-knowledge verifying keys}}
\newcommand{\joinSplitParameters}{\term{JoinSplit parameters}}
\newcommand{\JoinSplitParameters}{\titleterm{JoinSplit Parameters}}
\newcommand{\arithmeticCircuit}{\term{arithmetic circuit}}
\newcommand{\rankOneConstraintSystem}{\term{Rank 1 Constraint System}}
\newcommand{\primary}{\term{primary}}
\newcommand{\primaryInput}{\term{primary input}}
\newcommand{\primaryInputs}{\term{primary inputs}}
\newcommand{\auxiliaryInput}{\term{auxiliary input}}
\newcommand{\auxiliaryInputs}{\term{auxiliary inputs}}
\newcommand{\fullnode}{\term{full node}}
\newcommand{\fullnodes}{\term{full nodes}}
\newcommand{\anchor}{\term{anchor}}
\newcommand{\anchors}{\term{anchors}}
\newcommand{\UTXO}{\term{UTXO}}
\newcommand{\UTXOs}{\term{UTXOs}}
\newcommand{\block}{\term{block}}
\newcommand{\blocks}{\term{blocks}}
\newcommand{\header}{\term{header}}
\newcommand{\headers}{\term{headers}}
\newcommand{\blockHeader}{\term{block header}}
\newcommand{\blockHeaders}{\term{block headers}}
\newcommand{\Blockheader}{\term{Block header}}
\newcommand{\BlockHeader}{\titleterm{Block Header}}
\newcommand{\blockVersionNumber}{\term{block version number}}
\newcommand{\blockVersionNumbers}{\term{block version numbers}}
\newcommand{\Blockversions}{\term{Block versions}}
\newcommand{\blockTime}{\term{block time}}
\newcommand{\blockHeight}{\term{block height}}
\newcommand{\blockHeights}{\term{block heights}}
\newcommand{\genesisBlock}{\term{genesis block}}
\newcommand{\transaction}{\term{transaction}}
\newcommand{\transactions}{\term{transactions}}
\newcommand{\Transactions}{\titleterm{Transactions}}
\newcommand{\transactionFee}{\term{transaction fee}}
\newcommand{\transactionFees}{\term{transaction fees}}
\newcommand{\transactionVersionNumber}{\term{transaction version number}}
\newcommand{\transactionVersionNumbers}{\term{transaction version numbers}}
\newcommand{\Transactionversion}{\term{Transaction version}}
\newcommand{\coinbaseTransaction}{\term{coinbase transaction}}
\newcommand{\coinbaseTransactions}{\term{coinbase transactions}}
\newcommand{\CoinbaseTransactions}{\titleterm{Coinbase Transactions}}
\newcommand{\transparent}{\term{transparent}}
\newcommand{\xTransparent}{\term{Transparent}}
\newcommand{\Transparent}{\titleterm{Transparent}}
\newcommand{\transparentValuePool}{\term{transparent value pool}}
\newcommand{\deshielding}{\term{deshielding}}
\newcommand{\shielding}{\term{shielding}}
\newcommand{\shielded}{\textbf{\term{shielded}}}
\newcommand{\sheilded}{\textbf{\term{shielded}}}
\newcommand{\shieldedXTN}{\term{shielded} $ t \rightarrow z $ transaction}
\newcommand{\shieldedXTNs}{\term{shielded} $ t \rightarrow z $ transactions}
\newcommand{\shieldedNote}{\term{shielded note}}
\newcommand{\shieldedNotes}{\term{shielded notes}}
\newcommand{\xShielded}{\term{Shielded}}
\newcommand{\Shielded}{\titleterm{Shielded}}
\newcommand{\blockchain}{\term{block chain}}
\newcommand{\blockchains}{\term{block chains}}
\newcommand{\mempool}{\term{mempool}}
\newcommand{\treestate}{\term{treestate}}
\newcommand{\treestates}{\term{treestates}}
\newcommand{\nullifier}{\term{nullifier}}
\newcommand{\nullifiers}{\term{nullifiers}}
\newcommand{\xNullifiers}{\term{Nullifiers}}
\newcommand{\Nullifier}{\titleterm{Nullifier}}
\newcommand{\Nullifiers}{\titleterm{Nullifiers}}
\newcommand{\nullifierSet}{\term{nullifier set}}
\newcommand{\NullifierSet}{\titleterm{Nullifier Set}}
% Daira: This doesn't adequately distinguish between zk stuff and transparent stuff
\newcommand{\paymentAddress}{\term{payment address}}
\newcommand{\paymentAddresses}{\term{payment addresses}}
\newcommand{\viewingKey}{\term{viewing key}}
\newcommand{\viewingKeys}{\term{viewing keys}}
\newcommand{\spendingKey}{\term{spending key}}
\newcommand{\spendingKeys}{\term{spending keys}}
\newcommand{\payingKey}{\term{paying key}}
\newcommand{\transmissionKey}{\term{transmission key}}
\newcommand{\transmissionKeys}{\term{transmission keys}}
\newcommand{\keyTuple}{\term{key tuple}}
\newcommand{\notePlaintext}{\term{note plaintext}}
\newcommand{\notePlaintexts}{\term{note plaintexts}}
\newcommand{\NotePlaintexts}{\titleterm{Note Plaintexts}}
\newcommand{\notesCiphertext}{\term{transmitted notes ciphertext}}
\newcommand{\incrementalMerkleTree}{\term{incremental Merkle tree}}
\newcommand{\merkleRoot}{\term{root}}
\newcommand{\merkleNode}{\term{node}}
\newcommand{\merkleNodes}{\term{nodes}}
\newcommand{\merkleHash}{\term{hash value}}
\newcommand{\merkleHashes}{\term{hash values}}
\newcommand{\merkleLeafNode}{\term{leaf node}}
\newcommand{\merkleLeafNodes}{\term{leaf nodes}}
\newcommand{\merkleInternalNode}{\term{internal node}}
\newcommand{\merkleInternalNodes}{\term{internal nodes}}
\newcommand{\MerkleInternalNodes}{\term{Internal nodes}}
\newcommand{\merklePath}{\term{path}}
\newcommand{\merkleLayer}{\term{layer}}
\newcommand{\merkleLayers}{\term{layers}}
\newcommand{\merkleIndex}{\term{index}}
\newcommand{\merkleIndices}{\term{indices}}
\newcommand{\zkSNARK}{\term{zk-SNARK}}
\newcommand{\zkSNARKs}{\term{zk-SNARKs}}
\newcommand{\libsnark}{\term{libsnark}}
\newcommand{\memo}{\term{memo field}}
\newcommand{\taddr}{\textbf{\term{taddr}}}
\newcommand{\taddrs}{\textbf{\term{taddrs}}}
\newcommand{\zaddr}{\textbf{\term{zaddr}}}
\newcommand{\zksnarks}{\textbf{\term{ZK-SNARKs}}}
\newcommand{\zaddrs}{\textbf{\term{zaddrs}}}
\newcommand{\memos}{\term{memo fields}}
\newcommand{\Memos}{\titleterm{Memo Fields}}
\newcommand{\keyAgreementScheme}{\term{key agreement scheme}}
\newcommand{\KeyAgreement}{\titleterm{Key Agreement}}
\newcommand{\keyDerivationFunction}{\term{Key Derivation Function}}
\newcommand{\KeyDerivation}{\titleterm{Key Derivation}}
\newcommand{\encryptionScheme}{\term{encryption scheme}}
\newcommand{\symmetricEncryptionScheme}{\term{authenticated one-time symmetric encryption scheme}}
\newcommand{\SymmetricEncryption}{\titleterm{Authenticated One-Time Symmetric Encryption}}
\newcommand{\signatureScheme}{\term{signature scheme}}
\newcommand{\pseudoRandomFunction}{\term{Pseudo Random Function}}
\newcommand{\pseudoRandomFunctions}{\term{Pseudo Random Functions}}
\newcommand{\PseudoRandomFunctions}{\titleterm{Pseudo Random Functions}}

% conventions
\newcommand{\bytes}[1]{\underline{\raisebox{-0.22ex}{}\smash{#1}}}
\newcommand{\zeros}[1]{[0]^{#1}}
\newcommand{\bit}{\mathbb{B}}
\newcommand{\Nat}{\mathbb{N}}
\newcommand{\PosInt}{\mathbb{N}^+}
\newcommand{\Rat}{\mathbb{Q}}
\newcommand{\typeexp}[2]{{#1}\vphantom{)}^{[{#2}]}}
\newcommand{\bitseq}[1]{\typeexp{\bit}{#1}}
\newcommand{\byteseqs}{\typeexp{\bit}{8\mult\Nat}}
\newcommand{\concatbits}{\mathsf{concat}_\bit}
\newcommand{\listcomp}[1]{[~{#1}~]}
\newcommand{\for}{\text{ for }}
\newcommand{\from}{\text{ from }}
\newcommand{\upto}{\text{ up to }}
\newcommand{\downto}{\text{ down to }}
\newcommand{\squash}{\!\!\!}
\newcommand{\caseif}{\squash\text{if }}
\newcommand{\caseotherwise}{\squash\text{otherwise}}
\newcommand{\sorted}{\mathsf{sorted}}
\newcommand{\length}{\mathsf{length}}
\newcommand{\mean}{\mathsf{mean}}
\newcommand{\median}{\mathsf{median}}
\newcommand{\clamp}[2]{\mathsf{clamp\,}_{#1}^{#2}}
\newcommand{\Lower}{\mathsf{lower}}
\newcommand{\Upper}{\mathsf{upper}}
\newcommand{\bitlength}{\mathsf{bitlength}}
\newcommand{\size}{\mathsf{size}}
\newcommand{\mantissa}{\mathsf{mantissa}}
\newcommand{\ToCompact}{\mathsf{ToCompact}}
\newcommand{\ToTarget}{\mathsf{ToTarget}}
\newcommand{\hexint}[1]{\mathbf{0x{#1}}}
\newcommand{\dontcare}{\kern -0.06em\raisebox{0.1ex}{\footnotesize{$\times$}}}
\newcommand{\ascii}[1]{\textbf{``\texttt{#1}"}}
\newcommand{\Justthebox}[2][-1.3ex]{\;\raisebox{#1}{\usebox{#2}}\;}
\newcommand{\hSigCRH}{\mathsf{hSigCRH}}
\newcommand{\hSigLength}{\mathsf{\ell_{hSig}}}
\newcommand{\hSigType}{\bitseq{\hSigLength}}
\newcommand{\EquihashGen}[1]{\mathsf{EquihashGen}_{#1}}
\newcommand{\CRH}{\mathsf{CRH}}
\newcommand{\CRHbox}[1]{\SHA\left(\Justthebox{#1}\right)}
\newcommand{\SHA}{\mathtt{SHA256Compress}}
\newcommand{\SHAName}{\term{SHA-256 compression}}
\newcommand{\FullHash}{\mathtt{SHA256}}
\newcommand{\FullHashName}{\mathsf{SHA\mhyphen256}}
\newcommand{\Blake}[1]{\mathsf{BLAKE2b\kern 0.05em\mhyphen{#1}}}
\newcommand{\BlakeGeneric}{\mathsf{BLAKE2b}}
\newcommand{\FullHashbox}[1]{\FullHash\left(\Justthebox{#1}\right)}
\newcommand{\setof}[1]{\{{#1}\}}
\newcommand{\range}[2]{\{{#1}\,..\,{#2}\}}
\newcommand{\minimum}{\mathsf{min}}
\newcommand{\maximum}{\mathsf{max}}
\newcommand{\floor}[1]{\mathsf{floor}\!\left({#1}\right)}
\newcommand{\trunc}[1]{\mathsf{trunc}\!\left({#1}\right)}
\newcommand{\ceiling}[1]{\mathsf{ceiling}\left({#1}\right)}
\newcommand{\vsum}[2]{\smashoperator[r]{\sum_{#1}^{#2}}}
\newcommand{\vxor}[2]{\smashoperator[r]{\bigoplus_{#1}^{#2}}}
\newcommand{\xor}{\oplus}
\newcommand{\band}{\binampersand}
\newcommand{\mult}{\cdot}
\newcommand{\rightarrowR}{\buildrel{\scriptstyle\mathrm{R}}\over\rightarrow}
\newcommand{\leftarrowR}{\buildrel{\scriptstyle\mathrm{R}}\over\leftarrow}

\newcommand{\JoinSplit}{\text{\footnotesize\texttt{JoinSplit}}}

\newcommand{\affiliation}{\hairspace$^\dagger$\;}
\newcommand{\affiliationDuke}{\hairspace$^\ddagger$\;}

\begin{document}

\title{\doctitle \\
\Large \docversion}
\author{
\Large \leadauthor\hairspace\thanks{\; myhush.org, https://keybase.io/dukeleto, F162 19F4 C23F 9111 2E9C 734A 8DFC BF8E 5A4D 8019}
}
\date{\today}
\maketitle

\renewcommand{\abstractname}{}
\vspace{-8ex}
\begin{abstract}
\normalsize \noindent \textbf{Abstract.}

This paper will outline, for the first time, exactly how the "ITM Attack" (a linkability
attack against shielded transactions)
works against Zcash Protocol and how Hush is the first cryptocoin with a defensive
mitigation against it, called "Sietch". Sietch is already running live in production
and undergoing rounds of improvement from expert feedback. This is not an academic
paper about pipedreams. It describes production code and networks.

We begin with a literature review of all known metadata attack methods that can be
used against Zcash Protocol blockchains. This includes their estimated attack costs
and threat model. This paper then describes the "ITM Attack" which is a specific instance
of a new class of metadata attacks against blockchains which the author describes
as "Metaverse Metadata" attacks.

The paper then explains Sietch in detail, which was a response to these new attacks.
We hope this knew knowledge and theory helps cryptocoins increase their defenses
against very well-funded adversaries including nation states and chain analysis
companies.

A few other new privacy issues and metadata attacks against Zcash Protocol coins
will also be enumerated for the first time publicly. The ideas in this paper apply
to all cryptocoins which utilize transaction graphs, which is to say just about all
known coins. Specifically, the Metaverse Metadata class of attacks is applicable
to all Bitcoin source code forks (including Dash, Verge, Zerocoin and their forks),
CryptoNote Protocol coins (Monero and friends) and MimbleWimble Protocol (Grin, Beam, etc) coins
but these will not be addressed here other than a high-level description of how to apply
these methods to those chains.

\begin{quote}
In privacy zdust we trust.

If dust can attack us, dust can protect us.

    -- Sietch Mottos
\end{quote}

\vspace{2.5ex}
\noindent \textbf{Keywords:}~ \StrSubstitute[0]{\keywords}{,}{, }.
\end{abstract}

\vspace{-10ex}
\phantomsection
\addcontentsline{toc}{section}{\Large\nstrut{Contents}}

\renewcommand{\contentsname}{}
% http://tex.stackexchange.com/a/182744/78411
\renewcommand{\baselinestretch}{0.85}\normalsize
\tableofcontents
\renewcommand{\baselinestretch}{1.0}\normalsize
\newpage

\nsection{Introduction}

\nsection{Metadata Analysis of Zcash Protocol Blockchains: Basics}

\nsubsection{Concepts and Definitions}

This paper will be concerned with \textbf{transaction graphs}, which we define
in the traditional mathematical sense, of a set of nodes with a set
of vertices connecting nodes. In cryptocoins these always happen
to be directed graphs, since there are always funds which are unspent
becoming spent, i.e. a direction associated with each transaction.
This direction can be mathematically defined using the timestamp
of the transaction. Inputs are unspent at the time of the transaction
and outputs are spent at the time of the transaction.

There is a great deal of mathematical history devoted to the study
of \textbf{graph theory} that has not been applied to blockchain analysis,
mostly because there was no blockchains to analyze just a few years
ago and there was no financial profit in studying the data. That
has obviously drastically changed.

This paper will be primarily concered with \textbf{shielded transaction graphs}
which are \textbf{directed acyclic graphs (DAGs)}. A \shielded transaction
does not reveal the address of Alice, nor Bob, nor the amount transacted but it
does leak a large amount of metadata at the protocol level, which is not rendered
by block explorers nor well understood by the industry.

A \sheilded transaction has at least one \shielded address, referred to as a \zaddr.

We here concern ourselves only with \textbf{Zcash Protocol} which allows us to specify
a coherent language and symbols to describe the new ITM \zaddr linkability attack and mitigations
against it. All techniques here could technically also be used against transparent
blockchains, but since they leak all the useful metadata already, it would serve
no purpose. These new attacks can be thought of as "squeezing" new metadata leakage
from zaddrs out of places that nobody thought to look.

For those coins which only have a transaction graph at the network p2p level but
not stored on their blockchain (such as MimbleWimble coins), it does raise the bar
and attack cost. Since nation-states and are not cost-sensitive and obviously
have a vested interest to de-anonymize all blockchains, MW coins are not immune
to these new attacks being applied. A transaction graph still exists and so
the core concepts here can be applied.

\nsubsection{Types Of Shielded Transactions}

There are many types of shielded transactions, mirroring the complexity of transparent transactions
in Bitcoin Protocol. Here we introduce a convention for describing transactions.

\begin{itemize}

\item A fully shielded transaction $T$ with change $T:z \rightarrow (z,z)$
\item A fully shielded transaction $T$ with no change $T:z \rightarrow z$
\item A shielded transaction $T$ with transparent change $T:z \rightarrow (z,t)$
\item A dehielding transaction $T$ with change $T:z \rightarrow (t,z)$
\item A dehielding transaction $T$ with no change $T:z \rightarrow t$
\item A shielding transaction $T$ with no change $T:t \rightarrow z$
\item A shielding transaction $T$ with shielded change $T:t \rightarrow (z,z)$
\item A shielding transaction $T$ with transparent change $T:t \rightarrow (z,t)$
\end{itemize}

The above summarizes the most common transactions. Now say we want to describe a transaction which sends to 5 \zaddrs and 3 transparent addresses with no change:
$z \rightarrow z,z,z,z,z,t,t,t$ . To describe very large transactions subscripts can be used : $ z \rightarrow z_{52}, t_{39} $.

An individual transaction $T$ is a sub-graph of the full transaction graph $T \subset \mathbb{T}$ with vertex count of one.


\nsection{Metadata Analysis of Zcash Protocol Blockchains: Advanced}

\nsubsection{Active vs Passive Attacks/Analysis}

\nsubsection{Timing Analysis}

\nsubsection{Value Analysis}

\nsubsection{Fee Analysis}

\nsubsection{Input/Output Arity Analysis}

\nsubsection{Dust Attacks}

\nsubsection{Exchanges and Mining Pools}

\nsubsection{What does the explorer not show?}

\nsection{De-anonymization techniques literature review}

\nsubsection{Applications to new Shielded-only Chains}

\nsection{ITM Attack: z2z Transaction Linkability}

The \textbf{ITM Attack} specifically "attacks" a transaction $T:z\rightarrow z,z$, i.e.
a fully-shielded Zcash Protocol transaction which has the highest level of privacy.
First we describe the definition of the attack success, if any of the following
datums can be ascertained:

\begin{itemize}
\item  The value in the \zaddr sending funds.
\item  The value any of the \zaddrs receiving funds.
\item  The value of any ShieldedInputs spent in the transaction.
\item  A range of possible values being sent to any \zaddr, such as between 0.42 and 1.7 (with error estimate)
\item  A range of possible values stored in the sending \zaddr.
\end{itemize}

If any of the above metadata can be "leaked", the attack is a success. We note
that this attack is completely passive in it's core, but can be greatly improved
by adding active components "to taste". This is why metadata leakage attacks such
as this can be thought of a method of analysis or an outright attack.

The \textbf{ITM Attack} takes transaction id's and \zaddrs as input, or other OSINT which is readily available on Github, Twitter, Discord, Slack, public forms, mailing lists, IRC and many other locations. With these public resources, the \textbf{ITM Attack} can bridge the gap from theoretically interesting attack to actually de-anonymizing a \zaddr to it's corresponding social media accounts, email addresses, IP addresses, location data and more.

This attack is not for weekend warriors or individuals with small budgets and is not
cost-effective for attacking a single \zaddr. It's best suited for the largest
players in The Great Game, i.e NSA, GCHQ and friends. It's highly likely they already
utilize analysis and attacks described in this paper.

Only the most well-funded private
blockchain analysis companies will be able to afford the infrastructure for this
attack, but once the data is "mined" it is a commodity that can be bought and sold
to those with less resources.

The ITM is an additional "layer" of analysis that can be overload on top of all other
types of analysis, and in that way it has the potential to "finish" a lot of "partial
de-anonymizations", i.e. places where blockchain analysis provides some data, but not
enough to fully de-anon. When added to timing analysis, amount analysis and fee analysis,
it can identify that certain \zaddrs being involved in many transactions and their
approximate input and output values. This data is not available any other way and
exact values are not very important.

If a blockchain analyst can ascertain a transaction involves at least 1M USD in value
versus a few pennies of value, that directly the course of analysis and investigation.
Perfect de-anonymization is not needed and in practice does not matter. Software
enabled with data from ITM analysis will be able to identify transaction outputs as having certain ranges of values and potentially their associated zaddrs from OSINT data.

\nsubsection{ITM Attack: Assumptions}

Fully working example code is left as an exercise to the interested blockchain analysis company. We shall describe the attack in enough detail for experts to verify our claims and for developers to implement attacks and or defenses, in the spirit of radical transparency.

We assume an attacker has at least 100,000 USD in funds to dedicate to the operation of studying one particular Zcash blockchain. Most of this cost is in the purchase of a GPU/FPGA farm to crunch data. Blockchains with more history and larger shielded pools will be more costly to study.

We note that this attack is not financially feasible as a one-off, it's a methodology
to study an entire blockchain which can then be indexed and search for potentially valuabledata. Blockchain anlaysis companies and the IC are strategically positioned to use this
information with the least cost, since they already have massive infrastructure to support this new dataset.

\nsubsection{ITM Attack: Defeating \zksnarks}

We can think of this attack as a "defeat" of zero-knowledge mathematics only in
practice, not in theory. Many qualifications are needed. We in no way "broke"
the mathematics of \zksnarks, we are taking advantage of how \zksnarks are being
used in higher level protocols, i.e. the Zcash Transaction Format Protocol and
it's associated consensus rules.

So \zksnarks are sound and we have not actually leaked \textbf{knowledge} directly
from a \textbf{zero-knowledge proof}, that is mathematically impossible. We
have leaked knowledge from how these proofs are used in the larger system called
Zcash Protocol, itself an extension of Bitcoin Protocol which notoriously leaks
metadata.

\nsubsection{ITM Attack: Infrastructure}

This attack requires storing a lot of intermediate data in addition to the raw
blockchain data and data storage costs are likely the number two expense after
computing power. It is possible renting compute power can lower computing expenses
but will not lower data storage costs. If one is analyzing a blockchain of $ B bytes $
then a reasonable estimate is that $100*B bytes$ of intermediate storage will be needed
to analyze the data and then a highly compressed version of the final useful data
can likely be stored in $B/100 bytes$ or less. That is, the final datasize will be much
smaller than the input data but our intermediate will likely be two orders of magnitude
larger.

Assume we have a simulated blockchain at block $N$, held in stasis and the analyst
has their own mining hashrate to "push" the chain forward by it's own defined consensus
rules. This can be accomplished by blocking all outside nodes and only connecting
to the local hashrate.

We also assume the analyst can easily "spin up" a blockchain at a certain block height
and try a new change to extract new data. This is trivially possible with virtual
machine images, docker containers and/or Git, and is left as an exercise to the
motivated blockchain analyst.

\nsubsection{ITM Attack: Consensual Oracles}

We now analyze a specific $T: z \rightarrow z,z$ at a speficic block height $H$ which
defines a specific \textbf{shielded pool} containing unspent shielded outputs and their
associated metadata, such as \textbf{Merkle Tree} data.

Very specifically, the simulation will use the \textbf{SaplingMerkleTree} internal Zcash Protocol datastructure defined in src/zcash/IncrementalMerkleTree.hpp . The ITM Attack focuses on this data structure but others can and should be explored as metadata oracles, such as the \textbf{SaplingWitness} data.

At any given block height $H$ a shielded "note" or \textbf{zUTXO} is either spent or unspent. Just like transparent \textbf{UTXOs}, a \textbf{zUTXO} can be spent from the mempool, i.e. the output of a transaction in this block can be spent by another transaction.

Different implementations of Zcash Protocol may react differently to spending zfunds from the mempool and so that is definitely a potential area of research.

Known Sapling commitments/anchors are "swapped" into the SaplingMerkleTree one at a time,
in an attempt to identify if they are being spent. If the new solution tree is invalid, then the data that was added caused it to become an invalid tree for a particular reason and
that particular reason is conveniently given when consensus-level errors are emitted in Bitcoin and Zcash Protocols. These errors have their own error codes and provide a wealth of information leakage to the aspiring analyst. By trying various known bits of data and analyzing the exact consensus error codes emitted, information is leaked.

\nsection{Metaverse Metadata Attacks}

The ITM Attack is a special case of what we name \textbf{Metaverse Metadata Attacks}, applied
to Zcash Protocol shielded transaction graphs.

The term \textbf{Metaverse} is appropriate because alternate possible blockchain histories can be simulated to see what consensus rules would have produced. By meticulously changing
one piece of data at a time, the analyst can use the consensus rules at that moment in blockchain history as an \textbf{oracle}. In this sense, \textbf{Metaverse} attacks can be classified as \textbf{consensus oracle attacks}, similar to \textbf{compression oracle} attacks and \textbf{padding oracle} attacks such as BREACH and CRIME against TLS.

\nsection{Sietch: Theory}

The ITM Attack relies on the fact that the most common shielded transaction on most currently existing Zcash Protocol blockchains have only 2 outputs $T: z \rightarrow z,z$ and the basic fact that if some metadata can be leaked about one output, if it's spend or it's range of possible values, it provides a lot of metadata on the other output as well.

If there were 3 outputs, then there would be uncertainty involved, instead of a more direct algebraic relation such as "if one output had amount=5 then the other output had an amount of $total - 5$". When 3 \zaddr outputs are involved, knowing the value of one \zaddr output does not provide as much information on the value of any other particular \zaddr.

This principle obviously increases, as the number of outputs increases, the leakage of
the amount of any one \zaddr input becomes exceedingly less valuable and expensive
metadata to utilize.

\nsection{Sietch: Code In Production}

Sietch uses a default rule of a minimum of 7 \zaddr outputs in a transaction. Because
the average shielded transaction does not spend the input values exactly and there is
a change output, in practice the average Hush transaction has 8 \zaddr outputs.

This is currently not a consensus rule and only enforced at RPC layer. There are
currently various implementations of Sietch in our full node and lite wallets, which
use raw transactions.

\nsection{Advice To Zcash Protocol Coins}

TLDR: You probably want Sietch or something like it.

\nsection{Special Thanks}

Special thanks to jl777, ITM and denioD for their feedback. 

\nsection{References}

\begingroup
\hfuzz=2pt
\renewcommand{\section}[2]{}
\renewcommand{\emph}[1]{\textit{#1}}
\printbibliography
\endgroup

\end{document}
